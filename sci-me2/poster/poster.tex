\documentclass[12pt]{article}
\usepackage{amsmath, amssymb, amsfonts}
\usepackage{fullpage}
\usepackage{graphicx}


\begin{document}

\begin{figure}[h] 
\centering 
\includegraphics[scale=0.60]{ss.png}\\
\caption{Polysome profiles}
\end{figure} 




\section{Left Column (With IPTG Induction)}

These panels show cells grown with IPTG induction.
The presence of IPTG induces the overexpression of the DeaD-mCherry protein, resulting in the red fluorescence signals observed in these graphs. The red peaks represent the DeaD-mCherry helicase, indicating that the protein is being produced and can interact with ribosomal components, as evidenced by overlapping with green signals.

\subsection{Panel A (25°C, With IPTG)}

Peaks at 30S and 50S indicate the presence of small and large ribosomal subunits, respectively. The polysome peak, representing active translation, is less pronounced compared to control (no IPTG), suggesting some disruption in translation efficiency. The presence of significant red peaks (DeaD-mCherry) shows that DeaD-mCherry is overexpressed and forms condensates, which may indicate liquid-liquid phase separation (LLPS) activity impacting ribosome assembly or function.

\subsection{Panel C (37°C, With IPTG)}


Similar 30S and 50S peaks as in Panel A. The polysome peak is still reduced compared to no IPTG, indicating continued disruption of translation. The red peaks for DeaD-mCherry are prominent, suggesting that overexpression and potential LLPS activity are present at this higher temperature.

\subsection{Panel E (42°C, With IPTG)}

    
The 30S and 50S peaks are present, but the polysome peak is even more diminished, indicating severe disruption of translation at this high temperature. Red peaks of DeaD-mCherry are still visible, showing that overexpression occurs across all temperatures and may be exacerbating translation issues at higher temperatures.


\section{Right Column (Without IPTG Induction)}

These panels show cells grown without IPTG induction.
Without IPTG, the DeaD-mCherry gene is not expressed, so there are no red fluorescence signals in these graphs. Only the green peaks, representing the ribosomal protein L1-mAzami, are visible, indicating that ribosomal biogenesis and function are occurring normally without the overexpression of DeaD-mCherry.

\subsection{Panel B (25°C, No IPTG)}


Clear peaks at 30S and 50S subunits and a well-defined polysome peak indicating active and efficient translation. No red peaks, as DeaD-mCherry is not expressed without IPTG induction. This serves as a baseline for comparison, showing normal ribosome function and biogenesis.

 
\subsection{Panel D (37°C, No IPTG)}


Similar to Panel B with clear 30S, 50S, and polysome peaks. Efficient translation continues without IPTG induction. Absence of red peaks confirms no DeaD-mCherry expression.

    
\subsection{Panel F (42°C, No IPTG)}

Peaks for 30S, 50S, and polysomes are still present, indicating that translation efficiency remains high even at this elevated temperature without IPTG. No red peaks due to lack of DeaD-mCherry expression.


\section{Comparison}

\subsection{Polysome Peaks}

\subsubsection{With IPTG}
 
Polysome peaks are consistently reduced across all temperatures, indicating that DeaD-mCherry overexpression disrupts ribosome function and translation.

\subsubsection{Without IPTG}


Polysome peaks are well-defined and indicate robust translation, showing that normal ribosome biogenesis and function are maintained without DeaD-mCherry overexpression.


\subsection{Red Peaks (DeaD-mCherry Expression)}


\subsubsection{With IPTG}

Red peaks appear at all temperatures, confirming that DeaD-mCherry is overexpressed and likely forming condensates, potentially contributing to observed disruptions in ribosome function.

\subsubsection{Without IPTG}

No red peaks, as expected, indicating no overexpression of DeaD-mCherry, which correlates with normal ribosome function and efficient translation.

    
\subsection{Temperature Effects}

        
\subsubsection{With IPTG}

The disruptive effects of DeaD-mCherry overexpression are more pronounced at higher temperatures (37°C and 42°C), as evidenced by further diminished polysome peaks.

\subsubsection{Without IPTG}

Ribosome function and translation efficiency remain stable across all temperatures, underscoring the negative impact of DeaD-mCherry overexpression when induced by IPTG.


\end{document}
