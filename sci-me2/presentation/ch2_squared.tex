\documentclass[12 px]{article}

\usepackage{hyperref}
\usepackage{listings}
\usepackage{hyperref}
\usepackage{graphicx}
\usepackage{amsmath, amssymb, amsfonts}

\title{\huge Chi-Squared Test Results and Interpretations}


\begin{document}

\maketitle

\section*{HER2 Status and Metastasis}
\begin{lstlisting}[language=Python, breaklines=true]
import numpy as np
from scipy.stats import chi2_contingency

obs = np.array([[14, 1, 15], [4, 5, 9], [72, 4, 76]])

chi2, pval, dof, exp = chi2_contingency(obs)

print("Chi-square statistic:", chi2)
print("p-value:", pval)
        
\end{lstlisting}

\textbf{Objective:} To determine if there is a significant association between HER2 status and the presence of metastasis in the liver and other organs.\\[5px]

\textbf{Data}:

- Observed counts:

  - Liver Metastasis: [14 HER2+, 1 HER2-, Total: 15]
  
  - Other Organs Metastasis: [4 HER2+, 5 HER2-, Total: 9]
  
  - No Metastasis: [72 HER2+, 4 HER2-, Total: 76]
  

\textbf{Chi-Square Test Results:}

- Chi-square statistic: 22.833

- p-value: 0.000137


\textbf{Interpretation:}

The chi-square test of independence was performed to evaluate whether there is a 
statistically significant association between HER2 status (positive or negative) 
and the presence of metastasis (in the liver, other organs, or no metastasis).


\textbf{Chi-Square Statistic:}

The chi-square statistic of 22.833 indicates the degree of deviation between the observed frequencies and the expected frequencies under the null hypothesis of independence. 

\textbf{p-Value:}
 
The p-value of 0.000137 is extremely low, well below the common significance level threshold of 0.05. 

\textbf{Conclusion:}

Given the p-value is significantly less than 0.05, we reject the null hypothesis. This provides strong evidence that there is a statistically significant association between HER2 status and the presence of metastasis.
            
\textbf{Summary:}

The chi-square test results suggest that the HER2 status of patients is significantly
associated with the occurrence of metastasis. Specifically, the likelihood of having 
metastasis (whether in the liver or other organs) differs depending on whether the
HER2 status is positive or negative. This significant association implies that HER2 
status could be an important factor in understanding and potentially predicting metastasis patterns in patients.





\section*{HER2 Status and Patient Vital Status}
\begin{lstlisting}[language=Python, breaklines=true]
import numpy as np
from scipy.stats import chi2_contingency

observed = np.array([[82, 7], [8, 3]])
chi2, p, dof, expected = chi2_contingency(observed)

print("Chi-squared test statistic:", chi2)
print("p-value:", p)
        
\end{lstlisting}

\textbf{Objective:} To determine if there is a significant association between patient vital status (alive or deceased) and HER2 status (positive or negative).\\[5px]

\textbf{Data}:

- Observed counts:

   - Alive: [82 HER2+, 7 HER2-]
   
   - Deceased: [8 HER2+, 3 HER2-]

  

\textbf{Chi-Square Test Results:}

- Chi-square statistic: 2.224

- p-value: 0.1358


\textbf{Interpretation:}

The chi-square test of independence was conducted to assess the relationship between HER2 status and patient vital status.


\textbf{Chi-Square Statistic:}

The chi-square statistic of 2.224 reflects the difference between the observed frequencies and the frequencies expected under the null hypothesis (which assumes no association between the variables).

\textbf{p-Value:}
 
The p-value of 0.1358 is greater than the common significance level threshold of 0.05.

\textbf{Conclusion:}

Since the p-value is greater than 0.05, we fail to reject the null hypothesis. This indicates that there is no statistically significant association between HER2 status and patient vital status at the $5\%$ significance level.
            
\textbf{Summary:}

The chi-square test results suggest that there is no significant association 
between HER2 status and the vital status of patients. This implies that, based 
on the given data, HER2 status does not appear to be a predictor of whether 
patients are alive or deceased. This lack of significant association means 
that other factors might be more influential in determining patient vital status.





\section*{HER2 Status and Tumor Advancement}
\begin{lstlisting}[language=Python, breaklines=true]
import numpy as np
from scipy.stats import chi2_contingency

obs = np.array([[5, 2], [85, 8]])
chi2, pval, dof, expected = chi2_contingency(obs)

print("Chi-squared statistic:", chi2)
print("P-value:", pval)
        
\end{lstlisting}

\textbf{Objective:} To determine if there is a significant association between HER2 status(positive or negative) and tumor advancement (advanced or not advanced).\\[5px]

\textbf{Data}:

- Observed counts:

  - Advanced tumor: [5 HER2+, 2 HER2-]
  
  - Not advanced tumor: [85 HER2+, 8 HER2-]

\textbf{Chi-Square Test Results:}

- Chi-square statistic: 1.092

- p-value: 0.296


\textbf{Interpretation:}

The chi-square test of independence was performed to assess whether there is 
a statistically significant association between HER2 status and tumor advancement.


\textbf{Chi-Square Statistic:}

The chi-square statistic of 1.092 indicates the degree of difference between the observed frequencies and the expected frequencies under the null hypothesis of independence.

\textbf{p-Value:}
 
The p-value of 0.296 is greater than the common significance level threshold of 0.05.

\textbf{Conclusion:}

Since the p-value is greater than 0.05, we fail to reject the null hypothesis. This indicates that there is no statistically significant association between HER2 status and tumor advancement at the $5\%$ significance level.
            
\textbf{Summary:}

The chi-square test results suggest that there is no significant association 
between HER2 status and whether a tumor is advanced or not. This implies that,
based on the given data, HER2 status does not significantly influence the 
likelihood of tumor advancement. Other factors might play a more critical 
role in determining tumor advancement, and further investigation is 
needed to identify those factors.





\section*{HER2 Status and Patient Age}

\begin{lstlisting}[language=Python, breaklines=true]
import numpy as np
from scipy.stats import chi2_contingency

observed = np.array([[73, 7], [17, 3]])
chi2, p_value, dof, expected = chi2_contingency(observed)

print("Chi-square statistic:", chi2)
print("P-value:", p_value)
      
\end{lstlisting}

\textbf{Objective:} To determine if there is a significant association between age 
          (less than 50 vs. greater than 50) and HER2 status (positive vs. negative).
          
          
\textbf{Data}:

- Observed counts:

  - Age < 50: [73 HER2+, 7 HER2-]
  
  - Age > 50: [17 HER2+, 3 HER2-]
  

\textbf{Chi-Square Test Results:}

- Chi-square statistic: 0.174

- p-value: 0.677

\textbf{Interpretation:}

The chi-square test of independence was performed to assess whether there 
is a statistically significant association between age groups and HER2 status.

\textbf{Chi-Square Statistic:}

The chi-square statistic of 0.174 indicates the degree of deviation between the observed frequencies and the expected frequencies under the null hypothesis of independence.


\textbf{p-Value:}
 
The p-value of 0.677 is substantially greater than the common significance level threshold of 0.05.

\textbf{Conclusion:}

Since the p-value is greater than 0.05, we fail to reject the null hypothesis. 
            This indicates that there is no statistically significant association between
            age and HER2 status at the $5\%$ significance level.
            
\textbf{Summary:}

The chi-square test results suggest that there is no significant association between 
the age of patients (whether they are younger than 50 or older than 50) and their 
HER2 status. This implies that HER2 status does not significantly vary between 
the two age groups in this dataset. Other factors may need to be considered 
to understand the variations in HER2 status among patients better.






\section*{HER2 Status and Diagnostic Age}

\begin{lstlisting}[language=Python, breaklines=true]
import numpy as np
from scipy.stats import chi2_contingency

obs = [[78, 7, 85], [12, 3, 15]]
chi2, pval, dof, exp = chi2_contingency(obs)

print(f"Chi-square statistic: {chi2:.4f}")
print(f"P-value: {pval:.4f}")
      
\end{lstlisting}

\textbf{Objective:} To determine if there is a significant association between diagnostic age 
(less than 50 vs. greater than 50) and HER2 status (positive vs. negative).
          
\textbf{Data}:

- Observed counts:

  - Age < 50: [78 HER2+, 7 HER2-, Total: 85]
  
  - Age > 50: [12 HER2+, 3 HER2-, Total: 15]
  

\textbf{Chi-Square Test Results:}

- Chi-square statistic: 1.9608

- p-value: 0.3752

\textbf{Interpretation:}

The chi-square test of independence was performed to assess whether there is a statistically 
significant association between diagnostic age groups and HER2 status.


\textbf{Chi-Square Statistic:}

The chi-square statistic of 1.9608 indicates the degree of difference between the observed frequencies and the expected frequencies under the null hypothesis of independence.

\textbf{p-Value:}
 
The p-value of 0.3752 is greater than the common significance level threshold of 0.05.


\textbf{Conclusion:}

Since the p-value is greater than 0.05, we fail to reject the null hypothesis.
            This indicates that there is no statistically significant association between
            diagnostic age and HER2 status at the $5\%$ significance level.
            
\textbf{Summary:}

The chi-square test results suggest that there is no significant association 
between the age at diagnosis (whether patients are younger than 50 or older than 50) 
and HER2 status. This implies that HER2 status does not significantly differ between
the two diagnostic age groups in this dataset. Other factors might be more critical
in determining HER2 status, and further investigation is needed to understand the
relationship better.



\section*{HER2 Status and Tumor Location}

\begin{lstlisting}[language=Python, breaklines=true]
import numpy as np
from scipy.stats import chi2_contingency

obs = np.array([[28, 7],[62, 3]])
chi2, pval, dof, exp = chi2_contingency(obs)

print("Chi-square test statistic:", chi2)
print("p-value:", pval)
      
\end{lstlisting}

\textbf{Objective:} To determine if there is a significant association between tumor location (rectum and sigmoid vs. colon) and HER2 status (positive vs. negative).

          
\textbf{Data}:

- Observed counts:

  - Rectum and Sigmoid: [28 HER2+, 7 HER2-]
  
  - Colon: [62 HER2+, 3 HER2-]
  

\textbf{Chi-Square Test Results:}

- Chi-square statistic: 4.396

- p-value: 0.036


\textbf{Interpretation:}

The chi-square test of independence was conducted to evaluate whether there is a 
statistically significant association between tumor location and HER2 status.


\textbf{Chi-Square Statistic:}

The chi-square statistic of 4.396 indicates the degree of deviation between the observed
frequencies and the expected frequencies under the null hypothesis of independence.

\textbf{p-Value:}
 
The p-value of 0.036 is less than the common significance level threshold of 0.05.


\textbf{Conclusion:}

Since the p-value is less than 0.05, we reject the null hypothesis. 
This indicates that there is a statistically significant association between 
tumor location and HER2 status at the $5\%$ significance level.
            
\textbf{Summary:}

The chi-square test results suggest that there is a significant association between tumor
location (rectum and sigmoid vs. colon) and HER2 status. This means that HER2 status
significantly varies depending on whether the tumor is located in the rectum and 
sigmoid or the colon. This significant association implies that tumor location 
could be an important factor in understanding and predicting HER2 status in patients. 
Further investigation is needed to explore the underlying reasons for this 
association and its implications for treatment and prognosis.






\end{document}