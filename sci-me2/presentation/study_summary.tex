\documentclass[12pt]{article}
\usepackage{hyperref}



\usepackage{hyperref}
\usepackage{graphicx}
\usepackage{amsmath, amssymb, amsfonts}

\title{\huge Investigating HER2 Exon 20 Mutations in Non-Hereditary Colorectal Cancer: Prevalence, Clinicopathological Significance, and Implications for Diagnosis and Treatment}


\begin{document}

\tableofcontents
\maketitle

\begin{figure}[h]
\includegraphics[scale=.9]{graphical\_abstract}\\
\caption{Graphical Abstract}
\end{figure}

\section{Abstract}
The human epidermal growth factor receptor 2, HER2, is a transmembrane tyrosine kinase protein implicated in various cancers, including colorectal cancer. This study aimed to determine the prevalence of HER2 exon 20 mutations in non-hereditary colorectal cancer patients and assess their associations with clinicopathological features, survival, and metastasis.
We analyzed 100 tissue samples from non-hereditary colorectal cancer patients, and the target fragment was amplified and sequenced using specific primers. Tumors were categorized according to their stage, ${T1-T4}$. Our findings revealed that ${10\%}$ of the patients harbored HER2+ mutations, with a male-to-female ratio of ${58\%}$ to ${42\%}$. The distribution of tumor stages was as follows: T1 ${(9\%)}$, T2 ${(39\%)}$, T3 ${(47\%)}$, and T4 ${(5\%)}$. The prevalence of HER2+ mutations was significantly associated with tumor location, ${p=0.036}$,  and metastasis, ${p=0.0001}$, but not with other examined factors. This study demonstrates the presence of HER2 exon 20 mutations in a subset of non-hereditary colorectal cancer patients and their significant associations with clinicopathological features, suggesting a potential role for HER2 in the initiation, progression, and diagnosis of colorectal cancer.
Further research with larger sample sizes is warranted to validate these findings and explore HER2 as a diagnostic and therapeutic target. Additionally, future studies should investigate the molecular mechanisms underlying the observed associations between HER2 mutations and clinicopathological features.


\section{Introduction}

The human epidermal growth factor receptor 2 (HER2) is a protein that plays a crucial role in the growth and proliferation of cells. It is a member of the HER/EGFR/ERBB family, which consists of four receptor tyrosine kinases: HER1 (EGFR, ErbB1), HER2 (ErbB2), HER3 (ErbB3), and HER4 (ErbB4),(figure 2). These receptors are structurally related and share common features, including an extracellular ligand-binding domain, a transmembrane domain, and an intracellular domain that can interact with various signaling molecules.

\begin{figure}[h]
\centering
\includegraphics[width=\textwidth]{her\_family}
\caption{Members of the HER Family (HER1-HER4)}
\end{figure}


HER2 is unique among the HER family members as it does not directly bind ligands. Instead, HER2 activation occurs through heterodimerization with other ERBB members or homodimerization when HER2 concentrations are elevated, such as in cancerous tissues. This dimerization leads to the autophosphorylation of tyrosine residues within the cytoplasmic domain of the receptors, triggering multiple signaling pathways, including MAPK, protein kinase C (PKC), and Signal transducer and activator of transcription(STAT), (figure 3). These pathways promote cell proliferation and inhibit apoptosis, emphasizing the need for tight regulation to prevent uncontrolled cell growth.

\begin{figure}[h]
\centering
\includegraphics[width=\textwidth]{hetero\_her}
\caption{Homo- and Heterodimerization of the HER2 Family.}
\end{figure}

Amplification or overexpression of the HER2 gene, ERBB2, located on the long arm of human chromosome 17 (17q12), has been implicated in the development and progression of several aggressive cancers. HER2 overexpression is most notably associated with breast cancer, where HER2-positive tumors are characterized by rapid growth and a propensity for metastasis. However, HER2 overexpression is not limited to breast cancer; it is also observed in other malignancies, including ovarian, stomach, and lung adenocarcinomas, as well as aggressive forms of uterine cancer such as uterine serous endometrial carcinoma. For instance, HER2 is overexpressed in approximately $7-34\%$ of gastric cancer patients and in $30\%$ of salivary duct carcinomas.

Given its significant role in oncogenesis, HER2 has become an important biomarker and therapeutic target. Targeted therapies, such as monoclonal antibodies and tyrosine kinase inhibitors, have been developed to specifically inhibit HER2 signaling pathways, leading to improved outcomes for patients with HER2-positive cancers.

\section{Colorectal Cancer (CRC)}

Colorectal cancer (CRC), also known as bowel cancer, colon cancer, or rectal cancer, originates in the colon or rectum, parts of the large intestine. The majority of CRC cases are attributed to aging and lifestyle factors, with a smaller proportion resulting from genetic disorders. Key risk factors for CRC include diet, obesity, smoking, and lack of physical activity. Dietary elements such as red meat, processed meat, and alcohol are known to increase CRC risk.

An important molecular feature in a subset of colorectal cancer (CRC) cases is the amplification of the HER2 gene. Approximately $3-4\%$ of colorectal cancer tumors exhibit HER2 amplification, with a higher incidence $(6-8\%)$ observed in patients with wild-type KRAS. HER2 amplification is also more frequently found in left-sided colon tumors compared to right-sided ones. This overexpression of HER2 is significant because it can influence the aggressiveness of the cancer and its response to certain treatments.\\[5px]

Most individuals with colorectal and breast cancer have normal levels of the HER2 protein, indicating HER2-negative status. However, about 1 in 5 cases are HER2-positive, which means their HER2 levels are unusually high.



\section{Why Exon 20 of HER2}

Exon 20 of the HER2 gene (ERBB2) is a focal point in many studies due to its critical role in the protein's kinase domain. Mutations in exon 20 are significant for several reasons:

\textbf{Activation Mutations:} Exon 20 contains regions essential for the kinase domain of the HER2 protein. Mutations in this exon can lead to constitutive activation of the kinase, resulting in continuous cell signaling that promotes cell proliferation and survival, contributing to oncogenesis. Notably, most of the mutations related to different types of cancer in the HER2 gene are located in the tyrosine kinase domain (exons 18-24), with $94\%$ of these mutations occurring in exon 20, typically as heterozygotes.\\[5px]
\textbf{Drug Resistance:} Specific mutations in exon 20, such as insertions or duplications, are associated with resistance to tyrosine kinase inhibitors (TKIs), making it crucial to identify these mutations for effective treatment planning.\\[5px]
\textbf{Prognostic and Predictive Value:} Mutations in exon 20 are linked to poor prognosis in various cancers, including CRC, breast cancer, and lung cancer. Analyzing this exon helps in stratifying patients based on risk and tailoring personalized treatment plans.

\section{TNM Classification System}

The TNM classification system is a widely used method for staging cancers based on three key components: Tumor (T), Nodes (N), and Metastasis (M). The stages of CRC are as follows:

\textbf{Stage I:} Small tumor (T1) with no lymph node involvement (N0) and no metastasis (M0).
\textbf{Stage II:} Larger tumor (T2 or T3) or involvement of nearby lymph nodes (N1) but no metastasis (M0).
\textbf{Stage III:} Larger tumor and/or more extensive lymph node involvement (N2 or N3) but no metastasis (M0).
\textbf{Stage IV:} Any tumor size or lymph node involvement with distant metastasis (M1).


\section{Methods}

\subsection{Data Collection}

The study population comprised 100 individuals undergoing treatment or in the diagnostic stage of non-hereditary colorectal cancer. Tissue samples were obtained via biopsy of colorectal tissue. The following variables were examined:

\begin{enumerate}
    \item \textbf{Age of Patient}: The age of each individual participating in the study.
    
    \item \textbf{Diagnostic Age}: The age at which colorectal cancer (CRC) was diagnosed.
    
    \item \textbf{Advanced Tumor}: Indicates whether the tumor is advanced or not (\textit{True} or \textit{False}).
    
    \item \textbf{Differentiation}: Refers to the degree of differentiation of the tumor, categorized as \textit{high}, \textit{moderate}, or \textit{low}.
    
    \item \textbf{Medical History}: Indicates whether the individual has a medical history relevant to colorectal cancer (\textit{Yes} or \textit{No}).
    
    \item \textbf{Tumor Location}: Specifies the location of the tumor, either in the \textit{rectum \& sigmoid} or the \textit{colon}.
    
    \item \textbf{Metastasis}: Indicates whether metastasis has occurred (\textit{Yes} or \textit{No}).
    
    \item \textbf{Chemotherapy}: Specifies whether the individual has undergone chemotherapy treatment (\textit{Yes} or \textit{No}).
\end{enumerate}



\subsection{DNA Extraction and Purification}
DNA was extracted from samples using the Kiagene extraction kit, following the manufacturer's protocol. This process involved multiple centrifugation steps to ensure the efficient separation and purification of DNA. The quality and quantity of the extracted DNA were assessed using a spectrophotometer (Nanodrop), measuring absorbance at 260 nm and 280 nm.

\subsection{DNA Amplification}
To amplify a 549-nucleotide sequence in exon 20 of the HER2 gene, two specific primers (Figure 4. A) were employed. The conditions for the PCR amplification are detailed in Figure 4. B.

\begin{figure}[h]
\centering
\includegraphics[width=\textwidth]{pcr_results}\\
\caption{A) Primers Used for Amplifying a 549-Nucleotide Sequence in Exon 20 of the HER2 Gene. B) PCR Conditions for Amplifying a 549-Nucleotide Sequence in Exon 20 of the HER2 Gene.}
\end{figure}


\subsection{Gel Electrophoresis}
The quality of the amplified sequences was examined using agarose gel electrophoresis (Figure 5). This method allows for the visualization of the PCR products to confirm the presence and size of the amplified DNA fragments.

\subsubsection{Figure 5 Explanation}
\textbf{Lane M (Marker):} The 100 bp DNA ladder is used as a molecular size marker to estimate the size of the PCR products. The bands correspond to known DNA fragment lengths, with the 500 bp band prominently visible.
\textbf{Lanes 1-3:} These lanes contain the PCR products amplified using the HER2 primers. Each lane shows a clear band at approximately 549 bp, indicating successful amplification of the target sequence.
\textbf{Lane NC (Negative Control):} This lane contains the negative control, which shows no bands. This indicates that there was no contamination or non-specific amplification in the PCR setup.

\begin{figure}[h]
\centering
\includegraphics[scale=.9]{pcr_result}
\caption{Gel electrophoresis results. Marker (M): 100 bp DNA ladder; Lanes 1-3: PCR products for HER2 primer setup, showing expected band at 549 bp; NC: Negative control, showing no amplification.}
\end{figure}


\subsection{Sequencing}
The extracted and validated DNA samples were sequenced using an ABI Genetic Analyzer 3110 XL to determine the nucleotide sequence and identify any mutations present. The results of the sequencing are shown in Figure 6.

\begin{figure}[h]
\centering
\includegraphics[scale=.9]{sequencing}
\caption{Sequencing results. (A) Wild type sample without mutation, showing the expected nucleotide sequence. (B) Heterozygous mutated sample, where a cytosine (C) is changed to guanine (G), resulting in the codon change from CGC (arginine) to GGC (glycine).}
\end{figure}

\subsection{Statistical Analyses}
The data were analyzed using SPSS software. Quantitative variables were analyzed using the mean ± standard deviation, while qualitative variables were analyzed using percentages. A significance level of 0.05 was used for all statistical tests. In the descriptive analysis, frequencies and percentages were reported along with central tendency measures such as the mean. For the analytical analysis, the chi-square test was employed to assess associations between categorical variables.

\section{Results}
Analysis of the research findings reveals that among 100 samples of patients diagnosed with colorectal cancer, 10 individuals, constituting $10\%$, exhibited HER2 mutations. Notably, half of these mutations were detected in the early stages of the disease, while the remaining half were observed in stages 3 and 4. Regarding metastases, $25\%$ of patients displayed HER2+ metastases, while the remaining $75\%$ had HER2- metastases. Among those with HER2 positivity, $16.6\%$ manifested liver metastases, while $83.3\%$ displayed metastases in other organs.

Tumors were found to be situated in the colon in $30\%$ of the population, while $70\%$ were located in the rectum. Tumor differentiation analysis revealed that $80\%$ of cases displayed low to moderate differentiation, with the remaining $20\%$ exhibiting high differentiation. In terms of tumor advancement, $20\%$ of patients were deemed to have advanced tumors based on surrounding observations.

Based on the findings of this study, a significant relationship exists solely among survival indicators(patient vital status) $(P=0.043)$. Notably, there is a statistically significant correlation between metastasis $(P=0.0005)$, tumor location $(P=0.014)$, and the likelihood of malignancy among patient groups. This suggests a higher likelihood of polyp prevalence in the rectum. However, concerning the other indicators listed in figure 6 and their association with polyp prevalence and malignancy incidence, the differences were found to be insignificant $(P<0.05)$.

\begin{figure}[htbp]
\centering
\includegraphics[width=\textwidth]{results}\\
\caption{Exploration of Clinical, Demographic, and Pathological Variables Associated with HER2 Mutation Status}
\end{figure}




\begin{figure}[htbp]
\includegraphics[scale=.9]{analyse-results}\\
\caption{Demographic and Clinical Characteristics of Patients. (A) Comparison of the age distribution of patients, showing the number of patients aged above and below 50 years. (B) Comparison of the gender distribution of patients, showing the number of male and female patients. (C) Comparison of the stage of disease diagnosis, showing the number of patients diagnosed at stages 1, 2, 3, and 4.}
\end{figure}


\section{Conclusion and Suggestions}
The present study underscores the significance of HER2 gene mutation in the Iranian population as a potential key factor in colon cancer. Therefore, investigating the status of this gene could serve as a priority in the diagnostic process, preceding the examination of other genes, as suggested by prior research. 

Our findings revealed a heterozygous CG variant in exon 20 of the HER2 gene, resulting in a nucleotide alteration from A to G and a corresponding change in the amino acid sequence from arginine to glycine.

Although no significant associations were observed between age, gender, and size index with the prevalence of intestinal polyps and malignancy incidence in our study and reviewed literature, it's noteworthy that individuals over 50 years old exhibit a higher risk of developing invasive polyps and colorectal cancer. Additionally, the incidence of adenoma, invasive polyps, and colorectal cancer appears to be higher in men compared to women.

Moving forward, the following suggestions are proposed for further research:

\begin{enumerate}
    \item Conduct more extensive studies with larger sample sizes to validate the role of HER2 in the development, progression, or diagnosis of colorectal cancer.
    \item Investigate the intrinsic effects of anti-HER2 treatment on tumors with discordant gene expression in colorectal cancers through comprehensive studies.
    \item Explore HER2 expression using various methods, including Fluorescence In Situ Hybridization (FISH) and Immunohistochemistry (IHC), in different tissue and blood samples from colorectal cancer patients.
\end{enumerate}

\end{document}

